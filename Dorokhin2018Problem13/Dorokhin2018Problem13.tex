\documentclass[12pt,twoside]{article}
\usepackage{jmlda, amssymb, amsmath}
\begin{document}
\title
{Spherical CNN for QSAR prediction}
\author
{Dorokhin S.$^1$, Popova M.$^2$} % основной список авторов, выводимый в оглавление
\thanks
{Scientific advisor:  Strijov~V.\,V.
}
\organization
{$^1$Moscow Institute of Physics and Technology, $^2$University of North Carolina}
\abstract
{
    \textbf{Abstract:}
    The task of predicting molecular properties e.g. biological activity or solubility based on the atomic structure is called QSAR (quantitative structure-activity relationship) prediction.
    It is a classical problem in drug design.
    Despite various algorithms (quantile regression, radial basis function neural networks) are an acceptable solution, there is still a need for more precise models. 
    A model originally developed for 3D shapes recognition was chosen and put under carefull examination in context of QSAR forecasting.
    This model is Spherical CNN, first suggested by Taco S. Cohen et. al., who managed to demonstrate that this NN performs well in several applications, including atomization energy prediction.
    The implemented model is compared with common CNN, RNN, graph CNN and Random Forest.    
\bigskip
\newline
\textbf{Keywords}: \emph {QSAR prediction, Spherical CNN, drug design}
}
\maketitle
\section{Introduction}
    The idea of QSAR (Qualitive Structure Activity Realtionships) is to associate 2D or 3D structural representaion of a molecule with its biological or chemical properties. 
    This research is aimed at building an accurate QSAR prediction tool. There were several attempts to solve the problem. Meryam Zeryouh et. al. \cite{Wiener15} used graph representaion and suggested formulas for calculation of Wiener indices of complicated graphs. 
    Wiener indices do correlate with such properties as critical point (Stiel and Thodos \cite{Critical}) or viscosity (Rouvray and Crafford \cite{Visc}), but there is still no distinct relation to solubility or target activity, which are of extreme importance in drug design. 
    Nupur S Munjal et. al. \cite{Paclitaxel16} performed a non-linear multi-colinearity regression analysis to build model which predicts paclitaxel solubility. However, their model is designed for a specific compound and thus lacks universality. 
    Fatima Adilova and Alisher Ikramov \cite{MMP17} analyzed Matched Molecular Pairs (MMP) method in context of QSAR prediction. They managed to demonstrate that such an approach is inappropiate for QSAR modelling. 
\newline
    The method suggested in this article is based on Spherical Convolution Neural Networks. This concept was introduced by Taco S. Cohen et. al. \cite{SCNN}, who defined the correlation of two signals in SO(3) rotation group and the generalized convolution. 
    The unique feature of the CNN suggested in their article is that the abovementioned convolution allows to create a distortion-free projection of a spherical signal.
     Taco et. al. tested the CNN in various tasks, including prediction of atomization energies from molecular geometry. The model yielded excellent results and this makes applying it to QSAR prediction an interesting challenge. 
\newline
    The main drawback of the suggested model is its complexity resulting in significant number of parameters (around 1M) and huge memory and time resources required. 
    However, the resulting model is expcted to be a universal solution. It is compared with conventional CNN, RNN, graph CNN and Random Forest. \textcolor{red}{(The articles in the intro will soon be replaced)}
\section{Problem statement}
    Let $\mathbb{M}=\{\, m_i\mid i = \overline{1, n}\}$ be a set of molecules $m_i$, each described by 3D cartesian coordinates of all atoms it contains: $m_i=\{\,\mathbf{x}_j\in \mathbb{R}^3\mid j = \overline{1, k_i}\}$,
    where $k_i$ is the number of atoms in the molecule $m_i$.
    Every molecule $m_i \in \mathbb{M}$ has a certain property $y_i \in Y \subset \mathbb{R}$ associated with it, where $Y = \{\, y_1, y_2, ...\ y_n \}\ $ are molecular properties. 
    Their nature is not of great importance: it may be solubility, toxicity, bioactivity e. g.
\newline
    Let us consider a set of parametric models $\mathfrak{F}$ derived from convolutional neural networks class:\newline $\mathfrak{F} = \{\,f_i\colon(\mathbf{w}, m)\to \hat{y})\mid i \in \mathfrak{I}, m \in \mathbf{M} \}\ $, 
    where $\mathbf{w} \in W$  are parameters of a model and $\hat{y} \in \mathbb{R}$ is an estimated property.
    The task is to predict the property $y_i$ of a molecule $m_i$ based on its spacial structure only. It is considered to be a regression problem assuming $y_i \in N(\overline{y}, \sigma_y)$. Denoting the merit function as 
\begin{equation}
\label{merit}
    E(y, m, \mathbf{w}) = (y - f(m, \mathbf{w}))^2
\end{equation}
    the problem of training coefficients $\mathbf{w}$ could be represented by the following equation:
\begin{equation}
\label{argmin}
    \hat{w} = \underset{w \in W}{\argmin} \sum_{(y, m) \in (Y, M)} E(y, m, \mathbf{w})
\end{equation}
\bibliographystyle{plain}
\bibliography{Dorokhin2018Problem13}
\end{document}
