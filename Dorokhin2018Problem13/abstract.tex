\documentclass[12pt,twoside]{article}
\usepackage{jmlda}
\begin{document}
%\NOREVIEWERNOTES
\title
{Deep Learning for RNA secondary structure prediction}
\author
[Popova M, Nikitin P, Pastukhov S, Proshutinskiy D, Kurilovich A, Bazanov I, Nesterova I, Pikunov A, Dorokhin S.] % список авторов для колонтитула; не нужен, если основной список влезает в колонтитул
{Popova M, Nikitin P, Pastukhov S, Proshutinskiy D, Kurilovich A, Bazanov I, Nesterova I, Pikunov A, Dorokhin S.} % основной список авторов, выводимый в оглавление
\thanks
{Scientific advisor:  Strijov~V.\,V.
The problem given by:  Popova~M.\,?.
Consultant: Philipp~N.\,?.}
\organization
{MIPT}
\abstract
{
    \textbf{Abstract:}
        RNA secondary structure is an important feature which defines RNA functional properties. As secondary structure often defines functions, knowing RNAs secondary structure may help to investigate functions of novel RNA molecules. With current methods of RNA structure prediction only 80\% of structures can be accurately predicted. An AI-driven method for predicting RNA secondary structure inspired by neural machine translation model is proposed. The method is based on CONTRAfold which implements conditional log-linear models. Google's Neural Machine Translation System is used to map input RNN chains into output text. 
\bigskip
\textbf{Keywords}: \emph {Deep Learning, RNA structure, Machine Translation System}
}
\maketitle
\end{document}
