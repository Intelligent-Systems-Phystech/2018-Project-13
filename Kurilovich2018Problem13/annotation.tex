\documentclass[12pt,twoside]{article}
\usepackage{jmlda}
\begin{document}
\titleEng
    {Deep learning for RNA secondary structure prediction}
\authorEng
    {Popova M, Nikitin P, Pastukhov S, Proshutinskiy D, Kurilovich A, Bazanov I, Nesterova I, Pikunov A, Dorokhin S.}
\thanks
    {Scientific advisor:  Strijov~V.\,V.
The problem given by:  Popova~M.\,?.
Consultant: Philipp~N.\,?.}
\organizationEng
    {Moscow Institute of Physics and Technology (MIPT)}
\abstractEng{
RNA secondary structure is an important feature which defines RNA functional properties. Its importance can be illustrated by the fact, that it is evolutionary preserved and some types of functional RNAs always have the same secondary structure, for example all tRNAs fold into cloverleaf. As secondary structure often defines functions, knowing RNAs secondary structure may help investigate functions of novel RNA molecules. RNA folding is not as easy as DNA folding, because RNA is single stranded molecule which forms complicated base-pairing interactions, while DNA mostly exists as fully base paired double helices. Current methods of RNA structure prediction rely on experimentally evaluated thermodynamic rules, but with thermodynamics alone only 80\% of structures can be accurately predicted. We propose an AI-driven method for predicting RNA secondary structure inspired by neural machine translation model.

    \bigskip
    \textbf{Keywords}: \emph{RNA, secondary structure, deep learning}.}
    
\maketitle
\end{document}
